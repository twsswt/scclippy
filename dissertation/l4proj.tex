\pdfoutput=1

\documentclass{l4proj}

%
% put any packages here
%
\usepackage{graphicx}
\usepackage{hyperref}
\usepackage{float}
\usepackage{longtable}
\usepackage{listings}
\usepackage{color}

\definecolor{dkgreen}{rgb}{0,0.6,0}
\definecolor{gray}{rgb}{0.5,0.5,0.5}
\definecolor{mauve}{rgb}{0.58,0,0.82}

\lstset{frame=tb,
  language=SQL,
  aboveskip=3mm,
  belowskip=3mm,
  showstringspaces=false,
  columns=flexible,
  basicstyle={\small\ttfamily},
  numbers=none,
  numberstyle=\tiny\color{gray},
  keywordstyle=\color{blue},
  commentstyle=\color{dkgreen},
  stringstyle=\color{mauve},
  breaklines=true,
  breakatwhitespace=true,
  tabsize=3
}

\graphicspath{ {images/} }

\begin{document}
\title{Source Code Clippy (Scclippy) - IDE Micro-Task Recommender Plugin}
\author{Boris Nikolov}
\date{2015/2016}
\maketitle

\begin{abstract}
The Source Code Clippy plugin enables searching for code snippets from different sources and provides them to the developer. The main objective of the project is to build a plugin for an IDE that will support the user with searching and helpful related features, as well as giving them the ability to configure the system. There is a research interest in how beneficial excerpts of code are to the developers. Thus, the application's purpose is to provide tools with which the user can easily and intuitively interact. 
The system has been evaluated ... . The findings of the study show that ... . Feedback from users ... . Besides personal purposes, the plugin can be used ...
\end{abstract}

\educationalconsent
%
%NOTE: if you include the educationalconsent (above) and your project is graded an A then
%      it may be entered in the CS Hall of Fame
%
\tableofcontents
%==============================================================================

\chapter{Introduction}
\pagenumbering{arabic}

\section{Project context}

We have recently begun to explore a new area of research on micro-scale or end user software engineering. The goal of this project is to address the needs of end user software developers who are engaged in short term micro scale software development projects. Examples include the development of a script to configure an Arduino project; an excel macro for processing a spreadsheet of data or a python script to batch process a large number of pictures. Existing software engineering tools are not suitable for these projects because they are generally targeted at long term large scale efforts and require a considerable upfront investment. 

Various recommendation systems designed for software engineering purposes have been becoming more popular in recent years. Their purpose is to assist developers of all skills to find the information they need. This project is about building a tool that can help users to see usages of similar code by suggesting excerpts of code.

\section{Motivation}

The motivation for this project is to gain understanding of how useful code snippets are to developers. In addition, there are not many applications that provide searching for excerpts by retrieving data from different sources inside an IDE (Integrated Development Environment). This provides a chance to build such system, which will give the user the ability to choose between multiple options and be configured according to their preferences.

\section{Objectives}

A prototype code snippet recommender system for Sublime was developed by Tomasz Sadowski. This prototype continually monitors the contents of a developer's editor and searches Stack Overflow for similar snippets of code using an information retrieval engine (Terrier is used in the prototype). For example, a developer is working on a script to process text files and has started on the task of opening files. The recommender tool searches for relevant code snippets on Stack Overflow and presents them for auto insertion into the editor.

The goal of this project is to re-develop the prototype as a production quality plugin for an Integrated Development Environment such as Eclipse, IntelliJ, Visual Studio, Emacs or PyCharm. The plugin will help developers achieve small tasks and therefore should be highly usable so that the user can do their work seamlessly. In addition, the scope of the recommender's input should also be controllable, so that recommendations can be made on the basis of the currently typed command, or a selected range of text.

\section{Achievements}

The required plugin Source Code Clippy was successfully implemented as an IntelliJ IDE plugin for the Java programming language. It allows the user to search for code snippets in four different ways - by making requests to Stack Exchange excerpts API; by indexing files on disk; using a Web Service that has already indexed a collection of Stack Overflow documents; by performing Google search on the query. The plugin also offers useful to the user features - searching for code is done automatically when the user is typing as well as when the user selects part of their code; automatic insertion of code into the editor by double clicking on post and others.  
In terms of qualities, the plugin could be described as highly usable, efficient, customizable, and extensible.

\section{Dissertation structure}
The dissertation is structured in a chronological order. It starts with the background and requirements, follows the design and implementation process, and ends with the results of the evaluation and a discussion on the outcome of the project. The table below describes the report's structure.

\begin{table}[H]
\caption{Dissertation Structure}
\centering
\def\arraystretch{1.5}
\begin{tabular}{p{3cm}p{12cm}}
\hline
Chapter & Content \\
\hline
2. Background & A short description of the previous work related to the project. \\
3. Requirements & Discusses the gathering of the requirements, followed by a list of all requirements. \\
4. Design and Implementation & Describes the architecture of the plugin. Details the design decisions taken. Also contains a list of final design features.\\
5. Evaluation & Describes the test process and the user evaluation carried out, their results and a discussion of the results. \\
6. Conclusion: & Discusses the outcome of the project, future work possibilities and learning outcomes.  \\
\hline
\end{tabular}
\label{table:reportStructure}
\end{table}


\section{Terminology}
SCC refers to the Source Code Clippy (Scclippy) plugin.
\\
\\
RSSE is an abbreviation for Recommendation Systems for Software Engineering.

\chapter{Background}

\section{Related work}

In the previous years, a number of recommendation systems for Software Engineering have been developed to help developers with information and evaluation of alternative decisions. Most such systems have been focused on assisting while the users are in the process of programming.

There are many RSSE systems that use different input, as well as processing and visualisation techniques. Similar plugins have been outlined below.

\subsection{eRose}
The Eclipse IDE plugin eRose mines data from repositories such as Concurrent Versions System (CVS). It tracks consecutive changes in code and makes suggestions based on that information. The data is displayed in a separate view, after each save operation.

\begin{figure}[H]
\includegraphics[scale=0.4]{rose}
\centering
\caption{eRose plugin}\label{rose}
\label{fig:rose}
\end{figure}

\subsection{Strathcona}
Another example is the Strathcona system which extracts data about the structure of a code fragment in order to search a PostgreSQL database and use heuristics to rank the results. It shows a structural overview, as well as highlights based on the difference from the user's code. This system is particularly useful for frameworks that have been poorly documented.

\begin{figure}[H]
\includegraphics[scale=0.4]{strathcona}
\centering
\caption{Strathcona plugin}\label{strathcona}
\label{fig:strathcona}
\end{figure}

\newpage
\subsection{Suade}
Suade is also an Eclipse plugin which is specialised in automatically presenting suggestions by retrieving related elements (fields, methods). By design, it contains a dependency graph of the elements that can be interactively and iteratively updated by the developer.

\begin{figure}[H]
\includegraphics[scale=0.9]{suade}
\centering
\caption{Suade plugin}\label{suade}
\label{fig:suade}
\end{figure}

\subsection{CodeBroker}
CodeBroker uses as input the comments written by developers to assist them. A recommendation is produced every time a comment is written by analyzing the text and doing type-signature matching. The system also manages user-specific lists.

\begin{figure}[H]
\includegraphics[scale=0.4]{code-broker}
\centering
\caption{Code broker plugin}\label{code-broker}
\label{fig:code-broker}
\end{figure}

\subsection{Dhruv}
Dhruv is a system that recommends people and artifacts based on bug reports. It extracts data from the open source community (different types of users and content and their interactions) to create a Semantic Web, which later is used to match that to the bug report's information.

\begin{figure}[H]
\includegraphics[scale=0.45]{dhruv}
\centering
\caption{Dhruv plugin}\label{dhruv}
\label{fig:dhruv}
\end{figure}

\subsection{Expertise Browser}
Expertise Browser recommends people for a particular piece of code by considering the people who have modified the content in the past. However, an assumption is made that whoever made changes beforehand is an expert.

\begin{figure}[H]
\includegraphics[scale=0.4]{expertise}
\centering
\caption{Expertise Browser recommender}\label{expertise}
\label{fig:expertise}
\end{figure}

\subsection{ParseWeb}
ParseWeb is useful in situations when the developer needs an object to perform methods on, but cannot recall how to get the object. The tool crawls the web based on the user's input (object types) to search for recommendations.

\begin{figure}[H]
\includegraphics[scale=0.5]{parseweb}
\centering
\caption{ParseWeb recommender}\label{parseweb}
\label{fig:parseweb}
\end{figure}

\section{Evaluation of similar ideas}
Most systems and tools described above have something in common - they do recommendation by going through several stages - input handling, intermediate analyzing of some sort, and visualisation technique. However, they also have a unique approach that focuses on certain aspects. Existing code recommendation by the IDE's do not give example code, but rather offer small suggestions to the code. Source Code Clippy is special because it offers helpful features to assist the developer and mainly because it features different ways of searching for excerpts by focusing on getting data from Stack Overflow - one of the biggest sources of code snippets and a part of the Stack Exchange network of websites. Stack Overflow has more than 5 million users, which write 2.8 questions and 4.56 answers each minute (based on average statistics). So far the website has more than 11 million questions and 18 million answers in total.

\section{Description of Tomasz's prototype}
The Source Code Clippy plugin is a continuation of an already build prototype for Sublime which Tomasz Sadowski developed. The prototype was not evaluated. It comprises of a single feature that enables the user use the developer's editor and search Stack Overflow for similar snippets of code using an information retrieval engine (Terrier was used in the prototype). 

For example, a developer could be working on a script to process text files and has started on the task of opening files. The recommender tool searches for relevant code snippets on Stack Overflow and presents them for auto insertion into the editor. 

\chapter{Requirements}

\section{Requirements Elicitation and Gathering}
Requirements were gathered and discussed during informal weekly customer meetings.
The requirements and tasks were prioritised based on the MoSCoW approach (link). Github's issue tracker was used to list all of them and track their progress. The final list of all collected requirements is split into tables based on their priority and are also presented below.   (https://github.com/thrios/SourceCodeClippy/issues)

\subsection{Must Have}
Must have requirements are fundamental and crucial to the success of the project. Table ~\ref{table:mustTable} shows the details of the must-have requirements.

\begingroup
\renewcommand\arraystretch{1.5}
\begin{longtable}{p{2cm}p{4cm}p{9cm}}
\caption{Must Have Requirements}\\

\hline

FR & Requirement & Description \\
\hline
1 & Searching with local index & Based on some input, the user must be able to search using an index on the local file system, as well as create a new index/update existing one.\\
2 & Searching with Stack Exchange API & The user must be allowed to make requests to the Stack Exchange API based on a query and see the results returned\\
3 & Searching using a Web Service & The user must have the option to choose to search using a web service - a preindexed collection of Stack Overflow question and answer posts\\
4 & Searching with Google for the Stack Overflow domain & The user must be given the option to choose to search the text in the query pane with Google\\
5 & Configuration panel & The user must be provided with the functionality to change the settings of the plugin in order to configure and customize it to their needs - changing the local index used, the web service URI, etc\\
6 & Handling selection input & The user's mouse selection should be automatically\\
7 & Last statement typed query input & The user's key input should be captured and update the query input pane\\

\hline
\label{table:mustTable}
\end{longtable}
\endgroup

\subsection{Should Have}
The should have requirements are considered as important. Table ~\ref{table:shouldTable} represents all the 'should have' requirements along with their description.

\begin{table}[H]
\caption{Should Have Requirements}
\centering
\def\arraystretch{1.5}
\begin{tabular}{p{2cm}p{4cm}p{9cm}}
\hline
FR & Requirement & Description \\
\hline
8 & Insertion of code by double clicking a post & A user should be able to login with their Facebook account \\
9 & Filters of results & A user should be able to review the content before uploading, to delete any content that they do not wish to upload \\
10 & Feedback tab & The goal is to improve the system and therefore the user should be able to submit feedback \\
11 & Post useful meta data and customization & Posts should contain their type (question or answer), their score (if applicable) and a hyperlink leading to their source. There should be options to customize the post - let the user choose the text colour used, etc.\\
12 & Custom input pane listener & The user should be able to make search request on 'Enter' and new lines with the 'Shift' key when the focus is on the query pane\\
13 & Additional search request & The plugin should offer an additional search request to be made on the same query to allow more posts to be fetched and displayed\\
14 & Sorting results & The user should be able to sort results by relevance or by score\\
\hline
\end{tabular}
\label{table:shouldTable}
\end{table}

\subsection{Could Have}
Could have requirements are optional features the system could benefit from. Due to time constraints, FR21 was not implemented, but is considered for possible future work. Table ~\ref{table:couldTable}  lists the 'could have' requirements, as well as their descriptions.

\begin{table}[H]
\caption{Could Have Requirements}
\centering
\def\arraystretch{1.5}
\begin{tabular}{p{2cm}p{4cm}p{9cm}}
\hline
FR & Requirement & Description \\
\hline
15 & Number of calls to Stack Exchange API & The plugin could display the remaining calls that can be made to the API\\
16 & Highlighting code & Highlighting the text in HTML code tags to distinguish between code and text of a post\\
17 & Write a question & Functionality to enable the user to write a question on Stack Overflow\\
18 & Information tab & A tab for showing information to the user that they might need to know\\
19 & Default number of posts and max number of posts & Feature which gives the user the ability to change how many post they see after a search (on 'Enter'), and a secondary search (using the scroll)\\
20 & Saved settings & The options the user selects and changes could be saved automatically and loaded when the plugin starts\\
21 & Code Intelligent Search & The variable names and other terms in the query that are less relevant could have less weight attributed to them so that the results are more refined. The use of tags that allow the user to filter could be also a potential improvement.\\
\hline
\end{tabular}
\label{table:couldTable}
\end{table}

\section{Non-Functional Requirements}
Non-functional requirements represent the qualities the system is required to meet. Table ~\ref{table:nonFuncTable}  lists and describes the non-functional requirements for the plugin.
\begin{table}[H]
\caption{Non-Functional Requirements}
\centering
\def\arraystretch{1.5}
\begin{tabular}{p{2cm}p{4cm}p{9cm}}
\hline
NFR & Requirement & Description \\
\hline
1 & Intuitive interface & Users should not get confused and feel the plugin as something they are familiar with \\
2 & High responsiveness/performance in general & The user should not have to wait more than a few seconds \\
3 & High usability & Good user interface, good responsiveness, but also provide helpful features e.g. automating the process if possible and reasonable to do so \\
4 & Good documentation & Well-documented code allows for easy extension of the plugin \\
5 & Extensible & The system should allow for easy future extensions\\
\hline
\end{tabular}
\label{table:nonFuncTable}
\end{table}

\chapter{Design and Implementation}

\section{Use cases}
A list summarizing the different use cases follows.

\begin{itemize}
\item The user can update the query pane in various ways - directly, by mouse selection, or key input.
\item The user can choose to search from a number of sources.
\item The user can sort and filter results.

\item The user can execute a query
\begin{itemize}
\item The user can make another request that will provide more results.
\item The user can import code by double clicking on code from posts
\item The user can see meta data about each result/post
\end{itemize}

\item The user is allowed to view the past search history. 
\item The user can change the settings and any changes made are saved so they can be loaded on startup.
\item The user can see information about the plugin that will serve as a guide
\item The user can provide feedback and contact the developer if they need to
\end{itemize}



\subsection{Developer workflow}
Based on the use cases outlined, the typical developer workflow consists of typing a query by using the easiest way (might be mouse selection or automatically captured key input); searching for the result by changing the query in an iterative manner and the search method used appropriately until they find the result needed. Using the useful options provided (e.g. extended search, double click to import, etc) can speed up the process of completing the task being worked on.

In the UML Activity diagram \ref{fig:activity-diagram} an example workflow that uses a combination of use cases (which are coloured in orange) is demonstrated - the user opens the plugin, types something in the query pane, executes that query - if results are returned then the user can do an extra search for that query or double click on a code snippet; otherwise an error notification is shown. Finally, the user clicks on the 'Ask a question' because they have made sure they have not found an appropriate code snippet to the question or any explanation about the task. The process of interaction with the plugin ends intentionally when the user closes IntelliJ.

\begin{figure}[H]
\includegraphics[scale=0.5]{activity-diagram}
\centering
\caption{UML Activity diagram}\label{activity-diagram}
\label{fig:activity-diagram}
\end{figure}

\section{Iterations Overview}
Consecutive iterations of developing the plugin consisted of several stages:

\subsection{Familiarisation and preparation}
The first iterations consisted of familiarizing with how to develop a plugin for IntelliJ, how to do indexing and searching programmatically, and extracting data from the given PostgreSQL database. The database consists of a single table with the identifiers of posts and their data - such as parent id (for an answer that will be question id), score and others.

\subsection{Implementation of the plugin}
This stage consisted of building an initial version of the plugin that uses a local index, as well as Stack Exchange API search. Basic functionality such as code highlighting were also implemented.

\subsection{Expanding and improving the plugin}
The next iterations until the end of the project required developing a web service, automating existing features (input on selection and typing, copying code into editor, etc) and implementing new features such as adding configuration settings, search history, additional search options, feedback and information tabs.

\subsection{Evaluation}
Tests were implemented to test the functionality of the system. Various code metrics were used to determine the implementation quality of the plugin. User evaluation was carried out to better understand how useful the implemented features of the plugin are to developers in practice. 

\newpage
\section{Architecture}
Consecutive iterations of implementation caused the plugin's architecture to change. The figures \ref{fig:architecture-initial} and \ref{fig:architecture} use special colouring to differentiate different components - the logical parts of the system are coloured in red. The data sources can be identified by their green colour. The boxes in blue represent outgoing links.

\subsection{Initial architecture}

\begin{figure}[H]
\includegraphics[scale=0.4]{scc-architecture-initial}
\centering
\caption{Architecture diagram}\label{scc-architecture-initial}
\label{fig:architecture-initial}
\end{figure}

The project was initially required to use a local index on the PostgreSQL database given by Tomasz. A decision was made to export only the relevant data - Java questions and answers - and perform indexing on those posts. 

The process of extracting files from the initially provided database consists of the following steps:\\
1. Writing a query for retrieving questions and another one for answers, both of which add extra empty column for record separation.\\

\begin{lstlisting}
# Extracts questions
SELECT Posts.*, '' AS LineEnding
FROM Posts
WHERE Posts.tags LIKE '%<java>%'
\end{lstlisting}

\begin{lstlisting}
# Extracts answers
SELECT P1.*, P2.*, '' AS LineEnding
FROM Posts AS P1 JOIN Posts AS P2 ON P1.id = P2.parentid
WHERE P1.tags LIKE '%<java>%'
\end{lstlisting}

2. Exporting the results to a CSV file with a unique separator\\
3. Using a custom written CSV splitter to create the files used for indexing purposes

The plugin was also needed to retrieve code snippets by using Stack Exchange's API. Since the project is about retrieving code snippets, it was decided from the documentation that the plugin would use the 'excerpts' method provided. The response from all API methods is in the form of JSON objects.

\subsection{Final architecture}

\begin{figure}[H]
\includegraphics[scale=0.5]{scc-architecture}
\centering
\caption{Architecture diagram}
\label{fig:architecture}
\end{figure}

During the different iterations, it was clear that a local index was not a good enough option as it required every plugin user to have the files stored on disk. Therefore, a web service was created to replace the local index and replace the need for storage on each machine/client. However, it was seen that a local index could still be beneficial if the user wished to index a collection of their own documents. Thus, it was decided that the local index remained as a feature that provides a different and unique type of search.

The extraction of data from the database and extracting files from the CSV exported files required steps, which did not need to be made. Instead, it was discovered that indexing on the database could be done directly - this is a more straightforward way of indexing and minimizes the time taken to do the steps described which take a lot of time.

Searching with Google was added as an option to test how useful that would be to the user when related code snippets are involved.

To sum up, the user can use the IntelliJ Scclippy plugin to retrieve excerpts data from a number of data sources:

\begin{itemize}
\item by using a local index on the user's file system. The user is allowed also to create a new index or update the contents of existing one. 
\item by sending requests to the Web Service which in return will return a JSON response
\item by using the Stack Exchange API to make the request.
\item by clicking on a button that will open a browser window and search the query using Google for the domain of Stack Overflow. 
\end{itemize}

The system also has a hyperlink to each post returned, as well as to an online survey that the user can follow, fill and submit online. 

The client automatically saves its settings to a file relative to the plugin's installation in order to load them when a new IntelliJ client is started and the plugin was launched. Each instance of IntelliJ contains a separate instance of the Scclippy plugin.

\section{Design}

The code written was organized into two layers - business and presentation. The UML package diagram \ref{fig:scc-package-diagram} shows the design of the plugin by illustrating the different packages responsible for those layers as well as their composing packages. The business logic is handled in the package named 'plugin' and the presentation logic in the one named 'uicomponents'. All unnamed dependencies are of type "uses" - for example, the 'uicomponents.search' package uses the 'plugin.lucene' package because it needs the File class that was defined there - an import is needed so that the posts returned by searching are meaningful and can be displayed. The 'uicomponents.main' creates all tabs when the plugin is started. It also uses the 'uicomponents.search' package to update the input pane with the editor mouse selection string.

\begin{figure}[H]
\includegraphics[scale=0.5]{scc-package-diagram}
\centering
\caption{UML Package diagram}
\label{fig:scc-package-diagram}
\end{figure}

\section{User Interface}
The user interface was also changed frequently in order to keep up with the dynamic requirements and new features each iteration, but also due to ideas about improvements to make the look and feel of the plugin better. The differences of the basic and final design are shown below in terms of architecture and user interface.

\subsection{Sketches}

\begin{figure}[H]
\includegraphics[scale=0.2]{sketches}
\centering
\caption{Sketches of the proposed User Interface designs}\label{sketches}
\label{fig:sketches}
\end{figure}

In the beginning of this project, two sketches of how the plugin would look like were proposed. Both of them share some of the design choices made - the query pane to be located at the top with a button for searching (using Stack Overflow) immediately after. The designs differ mainly in how posts are displayed - the first keeps the whole posts one after another, while the latter splits the code from the text and the link to the post. The first design was chosen to be implemented as it is more simple and it might be disturbing for the user to look left and right for the information needed, especially if multiple code snippets, small or long in size, are involved. The latter design also could be problematic if the user has opened another tool window beside the one used for Scclippy - then the user is required to use a horizontal scroll to see the details of the post which is inconvenient. This is why whenever a new feature was added to the plugin, the position of that element was considered to be better if it were placed on the left side of the panel so that the user should not need to scroll to the right.

\subsection{Initial design}
\begin{figure}[H]
\includegraphics[scale=0.5]{ui-old}
\centering
\caption{Initial User Interface design}\label{ui-old}
\label{fig:ui-old}
\end{figure}

As a result, you can see that the first sketch was implemented successfully in the first iterations of the project. The code is highlighted on purpose so that the user is more aware of the presence of the code snippets in the post and their positions respectively.

\subsection{Final design}
The final user interface design has introduced several tabs to handle all the different features of the plugin. The next section discusses the design along with the features of the plugin.

\section{Final design features}
The plugin has its own closable window (ToolWindow) in IntelliJ that features five tabs. The function of each tab and its components is explained below.

\subsection{Search tab}
It contains four different types of search for excerpts - one for Google search with the domain of Stack Overflow (clickable by a button) and the rest three are selectable by a combo box:

\begin{itemize}
\item Local index
\item Web service (uses indexed collection of Stack Overflow Java documents)
\item Stack Exchange excerpts API (v2.2)
\end{itemize}

The search is performed when 'Enter' is typed on the content of an input field which the user can use to write a query. The input field can be used directly, as well as by selecting a code in the current project. The content is also automatically modified by capturing the user's types statements.

More results are also displayed when the scroll reaches the bottom - search is triggered if necessary.

The output of a query comes in forms of posts in the tool window each of
which contains posts from Stack Overflow. The code is highlighted based on code tags.

Along with the content, each post also displays a number of features in the bottom of the post:
The score\\
Type of post (question or answer)\\
The link to the Stack Overflow article is included (or the path to the file if local index is used)\\

For local index and web service search, formatted code insertion can be performed by double clicking on a code snippet in a post - if there are more than one code snippets in a post, the user will be prompt to choose which one.

When searching with Stack Exchange API, the number of calls left are displayed.

The default sorting used is by relevance for all search methods, however, the user could change to see the gathered results sorted by score.

A filter for results can be done by typing a number into a text box that will show only the posts with the minimum number of votes specified.

The plugin offers a feature to go to Stack Overflow by opening a new browser tab to ask a question in case the answer was not found.

\begin{figure}[H]
\includegraphics[scale=0.4]{tab-search}
\centering
\caption{Search tab, Darcula IntelliJ theme}
\label{fig:search-tab}
\end{figure}

\subsection{(Search) History tab}
The second tab is the search history tab, which enables the user to see query searches. Each new entry is displayed at the top. The total number of entries is 100. Currently, only recent query searches are shown - history is not saved to disk.

\begin{figure}[H]
\includegraphics[scale=0.4]{tab-history}
\centering
\caption{Search history tab, Darcula IntelliJ theme}
\label{fig:history-tab}
\end{figure}

\subsection{Settings tab}
The third tab is the 'Settings' tab that allows configuration of the plugin. The following options are included:\\

\begin{itemize}

\item General settings
\begin{itemize}
\item "Auto-resizable query input text area" (check box) - offers dynamically resizable query input area or static one that is long 5 rows and uses a scroll 
\item "Highlighted results" (check box) - can turn on/off highlighting of results
\item "Total number of posts" (number pane) - the number of posts after basic search using 'Enter' key
\item "Total number of posts after scrolling down search" (number pane) - controls the number of posts after extended search activated when scrolling to the bottom
\item "Colour of text" - text colour in posts chosen by typing its name or hex number (link)
\end{itemize}

\item Web service options
\begin{itemize}
\item "Web service URI" (key input pane) - for setting the URI to the chosen service
\end{itemize}

\item Local index options
\begin{itemize}
\item "Current index folder" - path to the index directory for local search
\item "Create/Update local index" - creates or updates an index by specifying index directory, data directory, and which of the two options.
\end{itemize}

\end{itemize}

\begin{figure}[H]
\includegraphics[scale=0.4]{tab-settings}
\centering
\caption{Settings tab, Darcula IntelliJ theme}
\label{fig:settings-tab}
\end{figure}

\subsection{Information tab}
The fourth tab is for plugin information which explains some details of the plugin - the purpose of the plugin, advice on which search method to use based on their advantages and disadvantages. At the bottom, the GitHub project link is added in case the user wishes to know more about the project.

\begin{figure}[H]
\includegraphics[scale=0.4]{tab-info}
\centering
\caption{Information tab, Darcula IntelliJ theme}
\label{fig:info-tab}
\end{figure}

\subsection{Feedback tab}
The fifth tab contains a participant consent form and a button to open a browser tab that leads to an online survey which the participant can use to submit feedback by completing the required fields and even answering the optional questions. 

The survey was created by using an already built web application -  smartsurvey.co.uk. There are benefits to this approach:
\begin{itemize}
\item Flexibility - the interface enables changes to the survey instead of creating a new one
\item Usability - it is easy to create a survey without having to code a new system
\end{itemize}

\begin{figure}[H]
\includegraphics[scale=0.4]{tab-feedback}
\centering
\caption{Feedback tab, Darcula IntelliJ theme}
\label{fig:feedback-tab}
\end{figure}

\section{Design decisions}

This section will discuss the different choices made throughout all iterations by comparing them using their pros and cons. The decisions taken aim to make the system meet the functional and non functional requirements in the best way possible. 

\subsection{Programming language and IDE choices}
Java was the programming language chosen and IntelliJ was picked due to their popularity - Java is frequently used among developers, while IntelliJ is also widely used and in addition offers very useful features to programmers - which were used even when developing the plugin.

\subsection{Java Swing vs Java FX}
The plugin's user interface was chosen to be implemented with the Java Swing toolkit rather than Java FX mainly because of familiarity reasons.

\subsection{Searching for excerpts}
Searching for code snippets with one method e.g. Stack Exchange API was not enough as it had its limitations and other ways that can provide complementing results existed. Therefore, the system offers a number of techniques for searching excerpts that have been outlined below:

\begin{itemize}

\item \textbf{Local index} - offers the user to freely create/update an index on a chosen data set. The major benefit of this approach is that the user can use it while being offline. The downside is that it requires documents, which the user might need, but does not have on their machine - if the user requires more information, about a topic, they may likely not have it.  

\item \textbf{Web Service} - has an already indexed collection of Stack Overflow posts for the Java programming language.  The benefit of the web service is that it covers a large dataset - about 2 million documents. In addition, it offers unlimited requests. The negative side is that it is not kept up to date like the Stack Exchange API. 
\\
\\
This method required creating a custom server that provides a RESTful Web Service that on requests performs search using Apache Lucene and returns only the number of documents specified (default is 5) that contain the id, body and score for each post. 
Initially, the server used Lucene to index the extracted files, however a decision was made to make the server use an index on the given PostgreSQL. This approach minimizes the steps needed to index the required posts.
\\
\\
The web service was deployed to an application server (rote.dcs.gla.ac.uk:9999). The major downside is that the it is only available to certain users which need to tunnel to the University's network and login with their credentials in order to have access. The server could have been deployed in the cloud to overcome this problem. However, different providers provide different solutions. For example, some of them do not offer PostgreSQL database support. Deploying the system itself has its own challenges - ideally, it should be done with minimal change of code. The plugin does not feature integration with cloud services. 

\item \textbf{Stack Exchange excerpts API} (link) - provides up to date Stack Overflow snippets, but each user has a limited number of requests, equal to 300 daily (by default). Another downside is that this approach depends on variable names e.g. in the worst case if the user chooses to search for a something with a very specific variable name then most likely they will not get any results back:
\\
\\
\textbf{String hsaudhqwudhyqgwd = "";}

\item \textbf{Google Search} - opens a browser tab that searches using the index created by Google on the Stack Overflow's domain. This is a generally good option since it is not so variable independent as the previous type of search.

\end{itemize} 

All of the above methods use an index to process requests quickly and thus make the plugin more usable. Since they use a different ranking mechanism, the result they return when queried will be different from the others.

\subsection{Automation of features}
\begin{itemize}

\item \textbf{Querying} - options include:
\begin{itemize}
\item Standard way - using a simple button
\item Automatic search listener which listens for any activity (typing, deleting or pasting text) which saves time – one less click per query - and is more intuitive.
\item Search on 'Enter' key pressed (faster than a button, slower than automatic search). Performed only when the user presses 'Enter'. It is faster than using a button, but slower than automatic search. Although automatic search seems the best, in practice, this option was implemented so as to avoid too many requests as there is a limit on the requests that can be made to Stack Overflow, and to avoid sending too many requests to the web service as they will not be handled at a sufficient pace.
\end{itemize}

\item \textbf{Typing} - when typing the input field is set to contain the text being typed. The text that is captured is triggered on special characters and the text of the last command range is restricted to the last seen semi colon or brace. This way only the relevant text being typed is considered and the approach also prevents ranges of text starting from import statements. The method does not restrict the user to use selection if they wish to capture a bigger piece of code.  

\item \textbf{Code insertion to query pane} - can be done in multiple ways:
\begin{itemize}
\item Standard copy paste
\item When selecting text the user could right click and select an option to allow him to set the text of the query (search begins automatically) 
\item Instant copy paste on highlight
The final options was chosen due to better usability and efficiency.
\end{itemize}

\end{itemize}

\subsection{Post size}
Currently, posts of all sizes are displayed. The size of posts could have a fixed number of rows and an added scroll. However, the latter idea was not regarded as a particularly good option. If the row number is low, the user would have to scroll. The problem with that is if the result is a single line long, then there would be empty space - a check for query size may be added to avoid that, but there is not much benefit in doing that. It is also more complex. The initial idea does not have drawbacks, therefore, it was implemented.

\subsection{Post content}
Since the input from the data sources is often in HTML, each post enables the rendering of HTML (v3.2). The implementation is done through JEditorPane. Originally JTextPane was used, however as it did not offer a lot of customization, it was  turned down. The usage of external libraries are another choice but in this case they were unnecessary.

Posts feature additional statistics:
\begin{itemize}
\item the score of the post
\item its type (question or answer)
\item the URI link
\end{itemize}

\subsection{Indexing and Full-Text Search}
Indexing was regarded as more suitable than Full-Text search to do the task of retrieving code snippets. The main reason is that indexing platforms are more efficient and flexible - they offer more customization and retrieval is done much faster.

\subsection{Lucene and Other retrieval platforms}
The choice of an open source indexing platform came down mainly from easiness of implementation. Terrier did not have a proper straightforward tutorial showing a programmatic way of calling the API. Apache Lucene Core (https://lucene.apache.org/core/) was used for indexing and retrieval of files. Since there were no problems, and the platform was sufficiently good for the plugin, other options were not considered. Each retrieval platform has its own implementation and principles, however, it should be pointed out that the decision made does not mean a less capable system was used. Apache Solr which uses Lucene could be considered as an alternative and an improvement in the future as it offers more features.

\subsection{Lucene settings}
\begin{itemize}

\item Stopwords were removed as most code is filtered out due to insufficient length and that is not what the plugin needs - for example, keywords are often short and should not be filtered.

\item No stemming is applied since retrieval is fast and there is no need to make the underlying dictionary small and sacrifice details. In terms of precision, the effectiveness of stemming were not compared to those without stemming. 

\item Pruning - neither document-based, nor term-based pruning was used for the same reason stemming was not used.

\item Special characters are escaped when querying to enable braces and other useful symbols 

\item Default settings for ranking were used (link):

\begin{itemize}
\item Frequent terms in a document increase the documents score
\item Popular terms have lower score
\item More terms that match the query means higher score
\item Short documents are preferred over long documents 
\end{itemize}

\end{itemize}

\subsection{Displaying more results for a query}
Additional search request is made (if necessary) when the scroll reaches the bottom of the plugin's tool window. Other options to be considered are: having a button instead (has to be clicked every time, less interactive); making search infinite (not very practical since only the top results are mostly being looked at).

\subsection{IntelliJ notifications} 
IntelliJ notifications are used to inform the user with the outcome of their actions. Examples include "Check connection to server. ..." which is displayed when the user is not connected to the Internet or other connection problems have occurred.

\subsection{Colour and IntelliJ themes}
The plugin provides the user with the option to change the colour of the text in posts. This is done so that the plugin is not strictly dependent on the theme used by the user. In addition, it enables the them to choose to configure the look of the plugin to match their subjective preferences. Users who have colour blindness are predicted to benefit from using this option.

\section{Challenges}
\begin{itemize}
\item Using Terrier was initially planned for indexing and querying (http://terrier.org/), however, due to the lack of examples and information on how to use its API programmatically, the use of Lucene was considered and used instead.

\item The lack of knowledge about Information Retrieval in the early stages of the project resulted in inability to make decisions about the in-depth information retrieval platform's options at that time

\item No previous experience with developing IntelliJ plugins slowed down the process of implementation

\end{itemize}

\chapter{Evaluation}

\section{Unit Testing}
The main searching functionality of the code was tested using JUnit. The classes which have a test case are those classes responsible for searching inside the editor:

\begin{itemize}
\item StackExchangeSearch
\item WebServiceSearch
\item LocalIndexedSearch
\end{itemize}

A major part of the code written is for the graphical user interface of the plugin. Testing that part of code is not currently supported by IntelliJ/JetBrains and therefore it was skipped. 

\section{Code metrics}

The MetricsReloaded plugin was used to extract code metrics for the project. https://plugins.jetbrains.com/plugin/93

\subsection{MOOD}

All code metrics from the MOOD collection have been used to determine the overall quality of an object-oriented project. Below are the results for each of the criteria used. Based on those metrics, it can be seen that the results in general can be described as good.

\begin{table}[H]
\small
\caption{MOOD code metrics}
\centering
\def\arraystretch{1.5}
\begin{tabular}{p{4cm}p{5cm}p{2.5cm}}
\hline
Metric & Short description & Result\\
\hline
Average hiding factor & ratio of the number of other classes a field is visible from & 0.75\\ 
Attribute Inheritance factor & ratio of what percentage of the fields for a class are due to inheritance and are not directly defined & 0.95\\
Coupling factor & the proportion of the classes coupled to other classes & 0.18\\
Method Hiding factor & the ratio of the number of classes a method is visible from & 0.35\\
Method Inheritance factor & the ratio of what percentage of the methods for a class are due to inheritance and are not directly defined & 0.04\\
Polymorphism factor & the probability that a given method will be overridden in a subclass & 3.11\\
\hline
\end{tabular}
\label{table:mood-codemetrics}
\end{table}

\subsection{Chidamber-Kemerer metrics}

Another set of metrics for judging the quality of the code is the Chidamber-Kemerer metrics. The results below show good results - response for class should not be above 50; number of children should be in the range of 0 to 10; WMC 1 to 50; DIT 2 to 5; CBO should be low; LCOM - generally good to be high, but may also mean a certain class may be doing too much.
http://www.virtualmachinery.com/sidebar3.htm
http://staff.unak.is/andy/StaticAnalysis0809/metrics/overview.html

\begin{table}[H]
\small
\caption{Chidamber-Kemerer code metrics}
\centering
\def\arraystretch{1.5}
\begin{tabular}{p{4cm}p{5cm}p{2.5cm}}
\hline
Metric & Short description & Result\\
\hline
Coupling between objects & number of classes and interface a class is coupled with & 5.28\\ 
Depth of inheritance tree & number of inheritance steps between the class and java.lang.Object & 2.64\\
Lack of Cohesion of methods & measures the cohesiveness of a class - higher values mean higher cohesion & 0.98\\
Number of children & number of direct subclasses & 0.06\\
Response for class & the number of methods in a class plus the number of methods that the class can access & 45.80\\
Weighted model complexity & cyclomatic complexity of each of the methods (in each class) & 3.11\\
\hline
\end{tabular}
\label{table:ck-codemetrics}
\end{table}

\subsection{Lines of code}
The lines of code were counted to have statistics about the code distribution. The counting includes comments as well, but excludes white spaces.

\begin{table}[H]
\small
\caption{Package code metrics}
\centering
\def\arraystretch{1.5}
\begin{tabular}{p{3cm}p{3cm}}
\hline
Client layer & Lines of code\\
\hline
Business & 795\\
Presentation & 1368\\
\hline
Total (from above) & 2162\\
\hline
\end{tabular}
\label{table:package-codemetrics}
\end{table}

Here is an overview from a pie chart that contains the different packages and their share of code:

\begin{figure}[H]
\includegraphics[scale=0.6]{code-distribution}
\centering
\caption{Code distribution for different packages}
\label{fig:code-distribution}
\end{figure}

\subsection{Language metrics}
A table is presented to give an idea of code distribution in terms of lines and files written for each language.

\begin{table}[H]
\small
\caption{Language code metrics}
\centering
\def\arraystretch{1.5}
\begin{tabular}{p{2.5cm}p{2.5cm}p{2.5cm}}
\hline
Language & Lines of code & Files\\
\hline
Java & 2210 & 34\\
XML & 996 & 10\\
\hline
\end{tabular}
\label{table:language-codemetrics}
\end{table}

\subsection{Javadoc coverage}
It was shown that 82.35 percent of the classes have been covered by Javadoc. The classes that have not been covered yet are straightforward implementations of features - like the info tab, the feedback tab, etc. 

\section{Acceptance testing}

\subsection{Technique}
The evaluation with participants was decided to be carried out by giving the plugin to users who could make their own choice about whether to fill in a survey or not. No time constraints were posed since it is considered best to evaluate the results of free interaction - just like developers would naturally use the plugin in practice. 

This approach works best if longer period of time is given to users. However, ethics approval for conducting experiments was received more than a month after applying for it. Despite that the process of evaluation remained unchanged because there is no negative effect in the way the evaluation is done, but is rather worse in terms of the time spend to test it. Although it is generally expected that the more testing is done, the better results the results are, it is good to keep in mind that it is not known whether the old participant would test again the new functionality/improvements being added.

\subsection{Participants' background}
The evaluation was aimed people who know Java or have started using it. The focus of finding participants was put on those who have access to the University network so that they can test searching with the Web service, which is one of the key features of the plugin. 

\subsection{User's view}
Users can participate in the evaluation of the plugin by going through the following steps:

1. Installing the plugin in IntelliJ (plugin was deployed to the JetBrains plugin repository)

2. Using the plugin's features for a desired period of time

3. Clicking on 'Feedback' tab

4. Reading the participant consent form

5. Clicking on a button, which will open the browser and take them to a survey

6. Answering at least all mandatory questions in the survey, and possibly even optional ones as well

7. Submitting the survey

\subsection{Survey questions and feedback}
In order to evaluate the plugin, a number of fields were asked to be filled by the participants (excluding the fields that the participant is required to agree so that he can participate). To summarize, the fields are aimed to cover the skills of the participant, the time of involvement with the plugin, the tasks performed, the satisfaction and dissatisfaction with the features and the reasons behind those two categories.

The fields and their possible answers (if applicable) are taken directly from the survey and are listed below. The '*' symbol denotes mandatory fields which must be completed.

\begin{itemize}
\item *State your Object oriented programming skills
\begin{itemize}
\item Beginner
\item Advanced/Expert
\end{itemize}

Most of the participants (21/22) stated that they are advanced or experts when using an Object Oriented programming language. This is not surprising considering that the participants contacted were colleagues.

\item *How long did you use the plugin for (e.g. 5 days total)?

Almost all participants used the plugin for a total time of 10 to 30 minutes (more in the next question). 

\item How often did you use the plugin (e.g. 5 minutes for 3 hours every day, 10 minutes for 1 hour 3 days a week)?

Most participants used the plugin for day. Only one of them used it for a 2-day period. As it was mentioned, the evaluation was limited in time due to approval being received so late.

\item What tasks did you perform?
\begin{itemize}
\item Bug fixing
\item Working on a new feature
\item Prototype experiment
\item Artificial task (trying out the tool)
\end{itemize}

In terms of tasks, the majority of participants (19 to be exact) aimed to try out the tool without using it for their project. Bug fixing was checked twice, another two ticks were for working on a prototype experiment, and the rest three on a new feature. From the above statistics, it is clear that some of the participants ticked more than one category. 

\item Will you continue using the plugin?
\begin{itemize}
\item Yes
\item No
\end{itemize}

It can be said that participants liked the plugin - 21 of 22 are willing to continue using it.

\item Give an example of a task that was easier when done with the plugin (If none state ‘none’).

The example tasks done were related to MapReduce, doing file I/O, object serialization, and other small ones like finding similar code that serves as an example for certain methods calls. Not all participants were willing to complete this field.

\item *What do you think about the plugin in general? What do you think about the user interface and the different ways of searching for snippets?

This is the most interesting question since it is an open question that gives the user the freedom to express themselves and share what they think about the plugin. 

To summarize, participants liked the plugin and its features. They were happy that they could do tasks in the IDE and that it saved them time. It seems most of them used the web service search and the Stack Exchange API. There is positive feedback on that there is a button for Google search that opens the browser and searches with the query. The user interface was considered as good, a few people recommended that it should be improved (see below).

\item What do you not like about the existing functionality of the plugin and why?

This question is very similar to the next one (which is about thoughts about new features) by having in mind that everything disliked can be improved. Therefore, this question is focused more on what can be changed, and the next question is about what can be added.

One of the participants mentioned that the default colour of the font is not appropriate for all IntelliJ themes. This issue needs investigation to see whether IntelliJ provides options to detect the colour theme used, and in case it does not, a setup window can appear the first time the plugin is loaded to allow the user to customize the colour. 

There was a complaint about the interface - it was suggested that there should be a button beside the query pane to make the search request since it is not obvious to everyone that pressing 'Enter' is faster and provides the same functionality. This can easily be implemented.

The arrow at the bottom of each snipper gives the impression it is clickable. The participant thought that details would appear for the particular post. There is a simple solution to this problem - the arrow can be replaced by something else.

A participant also mentioned that the search bar should always be at the top so that they can easily modify the search query without the need to scroll up so that the plugin is more usable. Implementation wise, this can be done without much effort - by having the scroll applied for the panel responsible for the posts instead of having that scroll the whole tab as it currently is.

Consecutive searches with the same query appear in history. This can be fixed making a check that will compare the last entry and the new one. Furthermore, a counter can be added to serve as an indication how many times a query is searched - perhaps the user switched the method of searching and

The 'Info tab' can contain more information about the plugin's features such as what can be double clicked and perhaps a more detailed description on how to search with an example.

\item Do you think there are any additional features that need to be implemented?

A participant recommended that there should be an option that allows automated removal of variable names since they are less meaningful than other code and are and the StackExchange API returns only results that match all terms in the query.

The history tab can have filters or text search in order to make the process of finding the desired query faster.
Include more information in the 'info tab' - for example, double clicking inserts a code snippet in the editor.

One of the proposals is to have support for languages other than Java. It is not something new - the improvement has been intended to be included in the plugin in the future.

Have a shortcut for searching with the query and maybe have other shortcuts as well

A nice feature suggested was allowing the user to mark all the results that the user has read as 'Read' or maybe even the plugin can automatically track the posts seen that the user has spend time on.

Some users are interested in the comments for a particular post so that they do not have to open the browser. This can be implemented by using the Stack Exchange API.

A participant has the idea to have automatic update on the query when a text in the post is selected - the plugin only does this for text in the editor. On the one hand, this is good because the user does not need to copy-paste the piece of code, however, on the other, this can delete the current search query. Nevertheless, it is possible to have this feature - by either providing a check box to enable/disable this functionality; by having a button that will allow the user to go back to the previous queries; or by making the automatic copy-paste to require a right click after selection where the option to paste the query will appear - the user would not need to move his mouse and paste the text.

One of the comments included how good code snippet insertion with web search is. The user wishes to be able to do the same for other types of search such as Stack Exchange API. Currently, the plugin uses special tags for the web service that indicate that a piece of code is there. In order to implement the new feature, the user should indicate where the snippet starts and ends. The best way to do that is to track mouse selection. However, a further action is needed by the user (maybe right clicking and then click on 'import into editor') so that text is not automatically included each time the user decides to use their mouse to select a given piece of text. 

Finally, a participants would like a new setting that will allow them to change to the size of the text for every post. It is possible to include this feature by modifying the CSS of each post just like highlighting is done currently.

\end{itemize}




\chapter{Conclusion}

\section{Project outcome}

Source Code Clippy is a configurable plugin that enables the search of excerpts from different sources, along with other helpful functionality.

The plugin was successfully implemented, tested, deployed and evaluated. All requirements were fulfilled, except one 'would-have' responsible for making search more sophisticated - adding tags and/or assigning less weight to variable names as they typically tend to be more irrelevant for a given query.

The evaluation showed that...

\section{Future work}

\subsection{Using the Cloud}
In the future, a cloud solution can be used for the web service so that it is accessible by everyone, while also being highly available. 

\subsection{Intelligent search}
It is reasonable to say that the plugin could benefit from having a more intelligent search which can be done using various techniques:

- More weight could be assigned to keywords and relevant code so that less important code such as variable names are not regarded as significant. 

- Making use of tags could prove to be beneficial by matching the user's needs - specifically when they would like to get results only from a subset of the data. Tags can be included for all three types of search (local index, web service, Stack Exchange API).

- Extracting other information from the provided database could help to include more data when returning results or for filtering purposes. For example, one or more of the following columns could be taken into account: 'creationdate', 'viewcount', 'answercount', 'commentcount', 'favouritecount'. The Stack Exchange excerpts API can also be used to its full potential. (http://api.stackexchange.com/docs/)

\subsection{Other possible extensions}
The plugin can be extended to have:

\begin{itemize}
\item All the improvements gathered through evaluation that were mentioned
\item Support for multiple languages
\item Caching of results (not much beneficial since retrieval is fast)
\item Storing functionality to save search history permanently
\item Options for better interaction with Stack Exchange - login/logout, notification when someone has answered the user's question
\end{itemize}

\section{Learning outcomes}

At the beginning of this project, there was a lack of experience on how to build an IntelliJ plugin and insufficient knowledge about information retrieval along with its according available platforms. However, by going through the process of creating the plugin, the required skills were learned. In addition, the existing skills of developing software were enhanced by making decisions about ideas and their implementation in practice.

\chapter{Bibliography}

%%%%%%%%%%%%%%%%
%              %
%  APPENDICES  %
%              %
%%%%%%%%%%%%%%%%
\begin{appendices}

\chapter{Chapter name}
\begin{verbatim}
\end{verbatim}

\end{appendices}

%%%%%%%%%%%%%%%%%%%%
%   BIBLIOGRAPHY   %
%%%%%%%%%%%%%%%%%%%%

\bibliographystyle{plain}
\bibliography{bib}

\end{document}
